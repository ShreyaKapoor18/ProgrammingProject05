Command Line Interface

\hypertarget{problem-statement}{%
\subsection{Problem Statement}\label{problem-statement}}

Check on this for a folder containing example images(PP5). These images
are taken from several open and free studies, articles or other sources.
Some of them have explicit or implicit meta data annotations. 1. Check
the meta data (EXIF, filename, \ldots{}) and write a concept which
metadata needs to be passed as additional information. 2. Your
application should take several command line parameters. --import should
take a image file name. --print is an option that prints all meta
information on the screen without doing anything more. Define -- and
doc- ument! -- the parameter --meta which takes additional meta
information. How should user pass these information? You can also split
this up in several parameters.

\hypertarget{description}{%
\subsection{Description}\label{description}}

Main Class: \textbf{CommandLineInterface.java} Subsidiary Class:
\textbf{Metadata.java}

\hypertarget{installing}{%
\subsection{Installing}\label{installing}}

In order to get the application running you will need to: 1. Git clone
the following repository

\begin{verbatim}
https://github.com/ShreyaKapoor18/biomed_images_metadb
\end{verbatim}

\begin{enumerate}
\def\labelenumi{\arabic{enumi}.}
\setcounter{enumi}{1}
\tightlist
\item
  Open Eclipse
\end{enumerate}

\begin{itemize}
\tightlist
\item
  Direct the workspace to biomed\_images\_metadb
\item
  Import existing maven project
\item
  Select the CLI (de.bit.pl02.task01) pom xml for importing the existing
  maven project.
\item
  Do a maven install
\end{itemize}

\begin{enumerate}
\def\labelenumi{\arabic{enumi}.}
\setcounter{enumi}{2}
\tightlist
\item
  After the application is installed you will see a
  task01-0.0.1-SNAPSHOT.jar in the /target folder
\end{enumerate}

\hypertarget{getting-started}{%
\subsubsection{Getting Started}\label{getting-started}}

\begin{enumerate}
\def\labelenumi{\arabic{enumi}.}
\tightlist
\item
  To execute created the jar use the following command
\end{enumerate}

\begin{verbatim}
java -cp ~/biomed_images_metadb/CommandLineInterface/CLI/target/task01-0.0.1-SNAPSHOT.jar  de.bit.pl02.pp5.task01.CommandLineInterface __\<options>__ __\<arguments>__ 
\end{verbatim}

Running the application with our pipeline 1. Get the path of the folder
that you want to be the directory, in our case PP5 2. Get the location
of the jar file from your local machine it should be something like
\texttt{\textasciitilde{}/biomed\_images\_db/CommandLineInterface/CLI/target/task01-0.0.1-SNAPSHOT.jar}

\begin{enumerate}
\def\labelenumi{\arabic{enumi}.}
\setcounter{enumi}{2}
\tightlist
\item
  Use it to insert the fileds in the Commands\_sample.txt file provided
  in this repository
\item
  Go to terminal, the path provided works best if you make sure you are
  in the base directory of your computer.
\item
  Use the command:
\end{enumerate}

\begin{verbatim}
bash ~/biomed_images_metadb/CommandLineInterface/CLI/Commands_sample.txt 
\end{verbatim}

\begin{itemize}
\tightlist
\item
  This shall execute all the commands in the .txt file on the command
  line
\end{itemize}

\hypertarget{application-output}{%
\subsection{Application Output}\label{application-output}}

The results from the Commands in the Commands.txt file can be seen in
the folder PP5 included in the repository. 1. You will be able to see
.meta files for all the image files in the directory you supplied (PP5
in our case). 2. All the meta files will contain fields * Author * Title
* Database * Infographic 3. For the files which already had a link in
the metadata, these fields have just ben appended to them

\hypertarget{command-line-interface}{%
\subsection{Command Line Interface}\label{command-line-interface}}

This application helps the user to create database objects and interact
with them. The user can provide a directory on which a database object
can be based. This directory should contain image files (.jpg, .jpeg or
.png) and corresponding metadata files (.txt) carrying the same file
name. The images are stored in the database along with the corresponding
metadata.

\textbf{Options}:\\
* -d or --directory : for specifying the path to the directory
containing the images

\begin{verbatim}
Example: 
-d  ~/PP5
\end{verbatim}

\begin{itemize}
\tightlist
\item
   -ip or --inputfile : the name of the input file you want to deal with
  \\
\end{itemize}

\begin{verbatim}
Example: 
-ip BError1.PNG
\end{verbatim}

\begin{itemize}
\tightlist
\item
   -p or --print : if you want to print the meta information or not \\
\end{itemize}

\begin{verbatim}
    -p option if you want to print otherwise omit
\end{verbatim}

\begin{itemize}
\tightlist
\item
   -m or --meta : if you want to add the meta information 
\end{itemize}

\begin{verbatim}
    -m option if you want to add meta information, omit otherwise
\end{verbatim}

\begin{itemize}
\tightlist
\item
   -im or --inputmeta : the values of metainformation you want to enter 
\end{itemize}

\begin{verbatim}
    -im Author,Title,Database_name,Infographic_number 
\end{verbatim}

\begin{verbatim}
* Author - String 
* Title - String 
* Database_name - String
* Infographic_number - Integers in range 1 to 4 i.e. (1,2,3,4)
    1. Implies image of a cell/tissue
    2. Implies image of a biological cartoon
    3. Implies that the image is a graph
    4. Implies the type of the image doesn't fit into the above classification 
\end{verbatim}

Mandatory commands needed for the application to run \\
* -d or --directory * -ip or --inputfile

\hypertarget{dependency-management}{%
\subsection{Dependency Management}\label{dependency-management}}

\hypertarget{prerequisites}{%
\subsubsection{Prerequisites}\label{prerequisites}}

The Java Version 1.8.0\_231 is used for this application. Apache Maven
Version 3.6.3 was installed from https://maven.apache.org/download.cgi.
Therefore the binaries apache-maven-3.6.3-bin.zip were downloded.

The following dependencies were added to Maven:

\textbf{JUnit} Version 3.8.1 \textbf{SQLite-JDBC} Version 3.18.0
\textbf{commons-cli} Version 1.4 \textbf{commons-io} Version 2.6
\textbf{commons-lang3} Version 3.4

The following plugins were used: * Maven jar plugin Version 1.4 * Maven
shade plugin Version 3.2.0

\hypertarget{built-with}{%
\subsection{Built With}\label{built-with}}

\begin{itemize}
\tightlist
\item
  \href{https://maven.apache.org/}{Maven} - Dependency Management
\end{itemize}

\hypertarget{authors}{%
\subsection{Authors}\label{authors}}

\textbf{Shreya Kapoor} \textbf{Sophia Krix} \textbf{Gemma van der Voort}

\hypertarget{acknowledgments}{%
\subsection{Acknowledgments}\label{acknowledgments}}

\textbf{Dr.~Jens Dorpinghaus} \textbf{Dr.~Sebastian Schaaf}
