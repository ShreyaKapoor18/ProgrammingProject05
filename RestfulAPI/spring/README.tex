Restful Web Service

\hypertarget{problem-statement}{%
\subsection{Problem Statement}\label{problem-statement}}

Your application should provide a RESTful web service using Spring, take
a look at https://spring.io/guides/gs/rest-service/ 1. Discuss, how
Spring and a a REST API works and how these two technologies work
together. 2. Provide at least the two functions described below. Write
documentation about what format you expect (JSON, XML, \ldots{}) and how
this API is working. 3./store should take an image and additional meta
data 4. /get should export a single image with meta-information or
should export a list of images matching the pattern.

\hypertarget{description}{%
\subsection{Description}\label{description}}

Main Class: \textbf{Task3Application.java}

Subsidiary Classes: 1. \textbf{FileController.java} 2.
\textbf{FileController.java} 3. \textbf{StoreImageResponse.java} 4.
\textbf{NotFoundExceptionDb.java}

\hypertarget{database-creator}{%
\subsection{Database creator}\label{database-creator}}

\hypertarget{installing}{%
\subsubsection{Installing}\label{installing}}

In order to get the application running you will need to: 1. Git clone
the following repository

\begin{verbatim}
https://github.com/ShreyaKapoor18/biomed_images_metadb
\end{verbatim}

\begin{enumerate}
\def\labelenumi{\arabic{enumi}.}
\setcounter{enumi}{1}
\tightlist
\item
  Open Eclipse
\end{enumerate}

\begin{itemize}
\tightlist
\item
  Direct the workspace to biomed\_images\_metadb
\item
  Import existing maven project
\item
  Select the de.bit.pl02.task03 pom xml for importing the existing maven
  project.
\end{itemize}

\begin{enumerate}
\def\labelenumi{\arabic{enumi}.}
\setcounter{enumi}{2}
\tightlist
\item
  Run Task3Application.java. in your IDE and go to the next section OR:
\item
  Do a maven install.
\item
  After the application is installed you will see a
  task03-0.0.1-SNAPSHOT.jar in the /target folder
\item
  To execute the jar use the following command:

  \begin{itemize}
  \tightlist
  \item
    Move to the spring directory using cd
  \item
    type: java -jar target/task03-0.0.1-SNAPSHOT.jar
  \end{itemize}
\end{enumerate}

\hypertarget{getting-started}{%
\paragraph{Getting Started}\label{getting-started}}

\begin{enumerate}
\def\labelenumi{\arabic{enumi}.}
\tightlist
\item
  Add the chosen database path and database name to the config.json
  file. Database name should be a unique identifier (DatabaseId).
\item
  Check if the file upload settings for your collaborators suit your
  needs in application.properties file, to be found in
  scr/main/resources. Default values are:
\end{enumerate}

\begin{itemize}
\tightlist
\item
  Threshold after which files are written to disk: 2KB
\item
  Max file size: 200 MB
\item
  Max Request Size: 215 MB
\end{itemize}

\begin{enumerate}
\def\labelenumi{\arabic{enumi}.}
\setcounter{enumi}{2}
\tightlist
\item
  Terminate the application using crtl+c. Collaborators now cannot
  access the databases anymore.
\end{enumerate}

\hypertarget{database-collaborator}{%
\subsection{Database Collaborator}\label{database-collaborator}}

No installation necessary.

\hypertarget{getting-started-1}{%
\subsubsection{Getting Started}\label{getting-started-1}}

\begin{enumerate}
\def\labelenumi{\arabic{enumi}.}
\tightlist
\item
  Ensure you have access to the network/server that the application is
  running on (eg http://localhost:8080/ or the address of your server).
\item
  copy the url given to you by the database owner into either your
  terminal or web browser.
\end{enumerate}

** Command line use (linux)** Use curl commands followed by the url, as
explained in the curl documentation: https://curl.haxx.se/docs/. In the
examples below, entries in curly brackets are to be filled in by the
user.

\begin{itemize}
\tightlist
\item
   /store : for storing the image and additional metadata. parameter
  file is required, the others are optional. 
\end{itemize}

\begin{verbatim}
Example:
curl -F author: {author name} -F title: {title} -F link: {link} -F file:@{filename including extension} http://localhost:8080/{databaseId}/store
\end{verbatim}

\begin{itemize}
\tightlist
\item
   /get : for getting image identifier and additional metadata. search
  can be done by author and/or title. Case sensitive. 
\end{itemize}

\begin{verbatim}
Examples: 
curl http://localhost:8080/{databaseId}/get?author={authorname}&title={title}
curl http://localhost:8080/{databaseId}/get?author={authorname}
curl http://localhost:8080/{databaseId}/get?title={title}
\end{verbatim}

\begin{itemize}
\tightlist
\item
   /get : for getting image in a .png format. search can be done by id. 
\end{itemize}

\begin{verbatim}
Examples: 
http://localhost:8080/{databaseId}/get?id={image id}
\end{verbatim}

\textbf{Web browser use} In the examples below, entries in curly
brackets are to be filled in by the user.

\begin{itemize}
\tightlist
\item
   /get : for getting image identifier and additional metadata. search
  can be done by author and/or title. Case sensitive. 
\end{itemize}

\begin{verbatim}
Examples: 
http://localhost:8080/{databaseId}/get?author={authorname}&title={title}
http://localhost:8080/{databaseId}/get?author={authorname}
http://localhost:8080/{databaseId}/get?title={title}
\end{verbatim}

\begin{itemize}
\tightlist
\item
   /get : for getting image in a .png format. search can be done by id. 
\end{itemize}

\begin{verbatim}
Examples: 
http://localhost:8080/{databaseId}/get?id={image id}
\end{verbatim}

\hypertarget{dependency-management}{%
\subsection{Dependency Management}\label{dependency-management}}

\hypertarget{prerequisites}{%
\subsubsection{Prerequisites}\label{prerequisites}}

The Java Version 1.8.0\_231 is used for this application. Apache Maven
Version 3.6.3 was installed from https://maven.apache.org/download.cgi.
Therefore the binaries apache-maven-3.6.3-bin.zip were downloded. Spring
Boot version 2.2.2 was installed from https://start.spring.io/.

The following dependencies were added to Maven:

\textbf{JUnit} Version 3.8.1 \textbf{SQLite-JDBC} Version 3.18.0
\textbf{commons-cli} Version 1.4 \textbf{commons-io} Version 2.6
\textbf{commons-lang3} Version 3.4

The following dependencies were added to Spring: \textbf{Rest
repositories} \textbf{Spring Data JPA}

The following plugins were used: * Maven jar plugin Version 1.4 * Maven
shade plugin Version 3.2.0

\hypertarget{built-with}{%
\subsection{Built With}\label{built-with}}

\begin{itemize}
\tightlist
\item
  \href{https://maven.apache.org/}{Maven} - Dependency Management
\end{itemize}

\hypertarget{authors}{%
\subsection{Authors}\label{authors}}

\textbf{Shreya Kapoor} \textbf{Sophia Krix} \textbf{Gemma van der Voort}

\hypertarget{acknowledgments}{%
\subsection{Acknowledgments}\label{acknowledgments}}

\textbf{Dr.~Jens Dorpinghaus} \textbf{Dr.~Sebastian Schaaf}
